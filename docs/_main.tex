% Options for packages loaded elsewhere
\PassOptionsToPackage{unicode}{hyperref}
\PassOptionsToPackage{hyphens}{url}
%
\documentclass[
]{book}
\usepackage{amsmath,amssymb}
\usepackage{lmodern}
\usepackage{iftex}
\ifPDFTeX
  \usepackage[T1]{fontenc}
  \usepackage[utf8]{inputenc}
  \usepackage{textcomp} % provide euro and other symbols
\else % if luatex or xetex
  \usepackage{unicode-math}
  \defaultfontfeatures{Scale=MatchLowercase}
  \defaultfontfeatures[\rmfamily]{Ligatures=TeX,Scale=1}
\fi
% Use upquote if available, for straight quotes in verbatim environments
\IfFileExists{upquote.sty}{\usepackage{upquote}}{}
\IfFileExists{microtype.sty}{% use microtype if available
  \usepackage[]{microtype}
  \UseMicrotypeSet[protrusion]{basicmath} % disable protrusion for tt fonts
}{}
\makeatletter
\@ifundefined{KOMAClassName}{% if non-KOMA class
  \IfFileExists{parskip.sty}{%
    \usepackage{parskip}
  }{% else
    \setlength{\parindent}{0pt}
    \setlength{\parskip}{6pt plus 2pt minus 1pt}}
}{% if KOMA class
  \KOMAoptions{parskip=half}}
\makeatother
\usepackage{xcolor}
\usepackage{longtable,booktabs,array}
\usepackage{calc} % for calculating minipage widths
% Correct order of tables after \paragraph or \subparagraph
\usepackage{etoolbox}
\makeatletter
\patchcmd\longtable{\par}{\if@noskipsec\mbox{}\fi\par}{}{}
\makeatother
% Allow footnotes in longtable head/foot
\IfFileExists{footnotehyper.sty}{\usepackage{footnotehyper}}{\usepackage{footnote}}
\makesavenoteenv{longtable}
\usepackage{graphicx}
\makeatletter
\def\maxwidth{\ifdim\Gin@nat@width>\linewidth\linewidth\else\Gin@nat@width\fi}
\def\maxheight{\ifdim\Gin@nat@height>\textheight\textheight\else\Gin@nat@height\fi}
\makeatother
% Scale images if necessary, so that they will not overflow the page
% margins by default, and it is still possible to overwrite the defaults
% using explicit options in \includegraphics[width, height, ...]{}
\setkeys{Gin}{width=\maxwidth,height=\maxheight,keepaspectratio}
% Set default figure placement to htbp
\makeatletter
\def\fps@figure{htbp}
\makeatother
\setlength{\emergencystretch}{3em} % prevent overfull lines
\providecommand{\tightlist}{%
  \setlength{\itemsep}{0pt}\setlength{\parskip}{0pt}}
\setcounter{secnumdepth}{5}
\usepackage{booktabs}
\ifLuaTeX
  \usepackage{selnolig}  % disable illegal ligatures
\fi
\usepackage[]{natbib}
\bibliographystyle{plainnat}
\IfFileExists{bookmark.sty}{\usepackage{bookmark}}{\usepackage{hyperref}}
\IfFileExists{xurl.sty}{\usepackage{xurl}}{} % add URL line breaks if available
\urlstyle{same} % disable monospaced font for URLs
\hypersetup{
  pdftitle={Data Analysis with R for Social Scientists},
  pdfauthor={Jakob Tures \& Jasper Tjaden},
  hidelinks,
  pdfcreator={LaTeX via pandoc}}

\title{Data Analysis with R for Social Scientists}
<<<<<<< Updated upstream
\author{Jakob Tures \& Jasper Tjaden}
=======
<<<<<<< HEAD
\author{Jasper Dag Tjaden}
=======
\author{Jakob Tures \& Jasper Tjaden}
>>>>>>> c05dcffc98789c802c90437072918e7ac1c0200f
>>>>>>> Stashed changes
\date{2023-08-04}

\begin{document}
\maketitle

{
\setcounter{tocdepth}{1}
\tableofcontents
}
\hypertarget{intro}{%
\chapter*{Intro}\label{intro}}
\addcontentsline{toc}{chapter}{Intro}

This course offers an accessible and easy introduction to one of the fastest growing statistical packages used in social science and data science more generally.

Please download the data used in the course \href{https://www.worldvaluessurvey.org/WVSDocumentationWV7.jsp}{here}. To find more about me, have a look at \href{https://jaspertjaden.com}{my website}. Also, feel free to watch me as I walk you through each lesson \href{https://www.youtube.com/playlist?list=PLr43hk2e3hFMg4tZdJsN0qzG5YkQB3A1c}{here}.

\textbf{Overview over the Course :}

\begin{itemize}
\tightlist
\item
  \textbf{\protect\hyperlink{intro-sem}{Week 1: Introduction to Seminar}}
\item
  \textbf{\protect\hyperlink{eda-1}{Week 2: Exploratory Data Analysis-I}}
\item
  \textbf{\protect\hyperlink{eda-2}{Week 3: Exploratory Data Analysis-II}}
\item
  \textbf{\protect\hyperlink{lin-t}{Week 4: Linear Regression-Theory}}
\item
  \textbf{\protect\hyperlink{lin-a}{Week 5: Linear Regression-Applied}}
\item
  \textbf{\protect\hyperlink{lin-e}{Week 6: Linear Regression-Exercises}}
\item
  \textbf{\protect\hyperlink{log-t}{Week 7: Logistic Regression-Theory}}
\item
  \textbf{\protect\hyperlink{log-a}{Week 8: Logistic Regression-Applied}}
\item
  \textbf{\protect\hyperlink{log-e}{Week 9: Logistic Regression-Exercises}}
\item
  \textbf{\protect\hyperlink{pm-t}{Week 10: Prediction or Margins-Theory}}
\item
  \textbf{\protect\hyperlink{pm-e}{Week 11: Prediction or Margins-Exercises}}
\item
  \textbf{\protect\hyperlink{report-v}{Week 12: Reporting and Visualizing}}
\item
  \textbf{\protect\hyperlink{dis-paper}{Week 13: Discussion of Ideas and term papers}}
\item
  \textbf{\protect\hyperlink{out-look}{Week 14: Outlook}}
\end{itemize}

\hypertarget{intro-sem}{%
\chapter{Introduction to Seminar}\label{intro-sem}}

All chapters start with a first-level heading followed by your chapter title, like the line above. There should be only one first-level heading (\texttt{\#}) per .Rmd file.

\hypertarget{a-section}{%
\section{A section}\label{a-section}}

All chapter sections start with a second-level (\texttt{\#\#}) or higher heading followed by your section title, like the sections above and below here. You can have as many as you want within a chapter.

\hypertarget{an-unnumbered-section}{%
\subsection*{An unnumbered section}\label{an-unnumbered-section}}
\addcontentsline{toc}{subsection}{An unnumbered section}

Chapters and sections are numbered by default. To un-number a heading, add a \texttt{\{.unnumbered\}} or the shorter \texttt{\{-\}} at the end of the heading, like in this section.

\hypertarget{eda-1}{%
\chapter{Exloratory Data Analysis - I}\label{eda-1}}

Here goes some texts.

\hypertarget{load-data}{%
\section{Load data}\label{load-data}}

Here goes some texts.

\hypertarget{introduce-wvs}{%
\section{Introduce WVS}\label{introduce-wvs}}

Here goes some texts.

\hypertarget{glimpse}{%
\section{glimpse()}\label{glimpse}}

Here goes some texts.

\hypertarget{skim}{%
\section{skim()}\label{skim}}

Here goes some texts.

\hypertarget{types-of-variables-skalen}{%
\section{Types of Variables/ Skalen}\label{types-of-variables-skalen}}

Here goes some texts.

\hypertarget{univaraite-statistics-means-sds-min-max}{%
\section{Univaraite statistics (means, SDs, min, max)}\label{univaraite-statistics-means-sds-min-max}}

Here goes some texts.

\hypertarget{ggplot}{%
\section{ggplot}\label{ggplot}}

Here goes some texts.

\hypertarget{histograms}{%
\subsection{Histograms}\label{histograms}}

Here goes some texts.

\hypertarget{boxplots}{%
\subsection{Boxplots}\label{boxplots}}

Here goes some texts.

\hypertarget{bar-graphs}{%
\subsection{Bar graphs}\label{bar-graphs}}

Here goes some texts.

\hypertarget{scatterplots}{%
\subsection{Scatterplots}\label{scatterplots}}

Here goes some texts.

\hypertarget{gtsummary}{%
\section{gtsummary}\label{gtsummary}}

Here goes some texts.

\hypertarget{kreuztabellen}{%
\subsection{Kreuztabellen}\label{kreuztabellen}}

Here goes some texts.

\hypertarget{eda-2}{%
\chapter{Exloratory Data Analysis - II}\label{eda-2}}

Here goes some texts.

\hypertarget{markdown-introduction}{%
\section{Markdown Introduction}\label{markdown-introduction}}

Here goes some texts.

\hypertarget{applying-edawvsown-data}{%
\section{Applying EDA(WVS/own data)}\label{applying-edawvsown-data}}

Here goes some texts.

\hypertarget{lin-t}{%
\chapter{Linear Regression \& DAGS - Theory}\label{lin-t}}

\hypertarget{introduction}{%
\section{Introduction}\label{introduction}}

\hypertarget{what-is-linear-regression}{%
\section{What is Linear Regression}\label{what-is-linear-regression}}

\hypertarget{when-and-for-what-can-it-be-used}{%
\section{When and for what can it be used?}\label{when-and-for-what-can-it-be-used}}

\begin{itemize}
\item
  Metric outcome variables
\item
  Predictors can be metric or categorical
\item
  Assumptions
\end{itemize}

\hypertarget{formal}{%
\section{Formal}\label{formal}}

\hypertarget{regression-formula}{%
\section{Regression Formula}\label{regression-formula}}

\begin{itemize}
\tightlist
\item
  Simple Linear Regression

  \begin{itemize}
  \tightlist
  \item
    show intercept + beta graphically
  \item
    How to get the coefficients

    \begin{itemize}
    \tightlist
    \item
      super short: OLS (graphically not how to solve); maybe leave out
    \end{itemize}
  \end{itemize}
\item
  Multiple Linear Regression

  \begin{itemize}
  \tightlist
  \item
    maybe also show graphically for 2 x (plane)
  \end{itemize}
\end{itemize}

\hypertarget{interpretation-of-results}{%
\section{Interpretation of results}\label{interpretation-of-results}}

\begin{itemize}
\tightlist
\item
  Show example from WVS data (without the R code)
\end{itemize}

\hypertarget{maybes}{%
\section{Maybes}\label{maybes}}

\begin{itemize}
\tightlist
\item
  Mediation
\item
  maybe theory into DAG session and example into application
\item
  Interactions

  \begin{itemize}
  \tightlist
  \item
    maybe introduce here and apply next session
  \item
    maybe push it all into next session
  \item
    or leave out completely
  \end{itemize}
\item
  Multiple outcomes
\item
  maybe leave out
\end{itemize}

\hypertarget{lin-a}{%
\chapter{Linear Regression - Applied}\label{lin-a}}

\hypertarget{example-research-question}{%
\section{Example research question}\label{example-research-question}}

\begin{itemize}
\tightlist
\item
  Short theory

  \begin{itemize}
  \tightlist
  \item
    DAG from this

    \begin{itemize}
    \tightlist
    \item
      What do we have to control for to identify effect of interest?
    \end{itemize}
  \end{itemize}
\end{itemize}

\hypertarget{application-with-wvs-data}{%
\section{Application with WVS data}\label{application-with-wvs-data}}

\hypertarget{r-code}{%
\section{R Code}\label{r-code}}

\begin{itemize}
\tightlist
\item
  lm()

  \begin{itemize}
  \tightlist
  \item
    formula syntax
  \end{itemize}
\end{itemize}

\hypertarget{interpretation-of-regression-table-in-practice}{%
\section{Interpretation of regression table in practice}\label{interpretation-of-regression-table-in-practice}}

\hypertarget{regression-diagnostics-relatively-short}{%
\section{Regression Diagnostics (relatively short?)}\label{regression-diagnostics-relatively-short}}

\hypertarget{mediation}{%
\section{Mediation}\label{mediation}}

\begin{itemize}
\tightlist
\item
  total + direct effect

  \begin{itemize}
  \tightlist
  \item
    use DAGs again
  \item
    apply with lm()
  \item
    Interpret results
  \end{itemize}
\end{itemize}

\hypertarget{maybes-1}{%
\section{Maybes}\label{maybes-1}}

\begin{itemize}
\tightlist
\item
  Interactions

  \begin{itemize}
  \tightlist
  \item
    In practice if introduced the week before
  \item
    Maybe also introduce and apply here
  \end{itemize}
\end{itemize}

\hypertarget{lin-e}{%
\chapter{Linear Regression - Exercises}\label{lin-e}}

Here goes some texts.

\hypertarget{application-of-linear-regression}{%
\section{Application of Linear Regression}\label{application-of-linear-regression}}

With WVS/own data: Students apply linear regression.

\hypertarget{log-t}{%
\chapter{Logistic Regression - Theory}\label{log-t}}

\hypertarget{introduction-1}{%
\section{Introduction}\label{introduction-1}}

\hypertarget{what-is-linear-regression-1}{%
\section{What is Linear Regression}\label{what-is-linear-regression-1}}

\begin{itemize}
\tightlist
\item
  Part of Generalized Linear Models
\end{itemize}

\hypertarget{when-and-for-what-can-it-be-used-1}{%
\section{When and for what can it be used?}\label{when-and-for-what-can-it-be-used-1}}

\begin{itemize}
\item
  Categorical outcome variables
\item
  Predictors can be metric or categorical
\item
  Assumptions
\end{itemize}

\hypertarget{formal-1}{%
\section{Formal}\label{formal-1}}

\hypertarget{regression-formula-1}{%
\section{Regression Formula}\label{regression-formula-1}}

\begin{itemize}
\tightlist
\item
  Logistic Regression

  \begin{itemize}
  \tightlist
  \item
    show logistic function (shape)
  \item
    Link function
  \item
    show intercept + beta graphically
  \item
    How to get the coefficients

    \begin{itemize}
    \tightlist
    \item
      super short: ML (graphically not how to solve); maybe leave out
    \end{itemize}
  \end{itemize}
\end{itemize}

\hypertarget{interpretation-of-results-1}{%
\section{Interpretation of results}\label{interpretation-of-results-1}}

\begin{itemize}
\tightlist
\item
  Show example from WVS data (without the R code)
\end{itemize}

\hypertarget{maybes-2}{%
\section{Maybes}\label{maybes-2}}

\begin{itemize}
\tightlist
\item
  Mediation
\item
  maybe to hard to do correctly
\item
  khb package available?
\item
  Interactions

  \begin{itemize}
  \tightlist
  \item
    similar problem to above
  \end{itemize}
\item
  Multiple outcomes
\item
  maybe leave out
\item
  Multinomial logistic regression
\end{itemize}

\hypertarget{log-a}{%
\chapter{Logistic Regression - Applied}\label{log-a}}

\hypertarget{example-research-question-1}{%
\section{Example research question}\label{example-research-question-1}}

\begin{itemize}
\tightlist
\item
  Short theory

  \begin{itemize}
  \tightlist
  \item
    DAG from this

    \begin{itemize}
    \tightlist
    \item
      What do we have to control for to identify effect of interest?
    \end{itemize}
  \end{itemize}
\end{itemize}

\hypertarget{application-with-wvs-data-1}{%
\section{Application with WVS data}\label{application-with-wvs-data-1}}

\hypertarget{r-code-1}{%
\section{R Code}\label{r-code-1}}

\begin{itemize}
\tightlist
\item
  glm()

  \begin{itemize}
  \tightlist
  \item
    family and link
  \end{itemize}
\end{itemize}

\hypertarget{interpretation-of-regression-table-in-practice-1}{%
\section{Interpretation of regression table in practice}\label{interpretation-of-regression-table-in-practice-1}}

\begin{itemize}
\tightlist
\item
  What are Logits?
\item
  Maybe: Odds Ratios
\end{itemize}

\hypertarget{regression-diagnostics-relatively-short-1}{%
\section{Regression Diagnostics (relatively short?)}\label{regression-diagnostics-relatively-short-1}}

\hypertarget{mediation-1}{%
\section{Mediation}\label{mediation-1}}

\begin{itemize}
\tightlist
\item
  total + direct effect

  \begin{itemize}
  \tightlist
  \item
    use DAGs again
  \item
    apply with glm()
  \item
    Interpret results
  \end{itemize}
\item
  But maybe leave out completely (see session 7)
\end{itemize}

\hypertarget{maybes-3}{%
\section{Maybes}\label{maybes-3}}

\begin{itemize}
\tightlist
\item
  Interactions

  \begin{itemize}
  \tightlist
  \item
    In practice if introduced the week before
  \item
    Maybe also introduce and apply here
  \end{itemize}
\end{itemize}

\hypertarget{log-e}{%
\chapter{Logistic Regression - Exercises}\label{log-e}}

Here goes some texts.

\hypertarget{application-of-logistic-regression}{%
\section{Application of Logistic Regression}\label{application-of-logistic-regression}}

With WVS/own data: Students apply linear regression.

\hypertarget{pm-t}{%
\chapter{Prediction or Margins - Theory}\label{pm-t}}

Here goes some texts.

\hypertarget{predicted-probabilities}{%
\section{Predicted probabilities}\label{predicted-probabilities}}

At various co-variate levels

\hypertarget{marginal-effects}{%
\section{Marginal Effects}\label{marginal-effects}}

\hypertarget{pm-e}{%
\chapter{Prediction or Margins - Exercises}\label{pm-e}}

Here goes some texts.

\hypertarget{application-of-regression}{%
\section{Application of Regression}\label{application-of-regression}}

With WVS/own data: Students apply linear+logistic regression from previous exercises.

\hypertarget{report-v}{%
\chapter{Reporting and Visualization}\label{report-v}}

Here goes some texts.

\hypertarget{formatted-regression-tables}{%
\section{Formatted regression tables}\label{formatted-regression-tables}}

Here goes some texts.

\hypertarget{publication-ready-formatting-labelling-of-visuals}{%
\section{Publication-ready formatting/ labelling of visuals}\label{publication-ready-formatting-labelling-of-visuals}}

Here goes some texts.

\hypertarget{coefficient-plots}{%
\section{coefficient plots}\label{coefficient-plots}}

Here goes some texts.

\hypertarget{dis-paper}{%
\chapter{Discussion of ideas and term papers}\label{dis-paper}}

Here goes some texts.

\hypertarget{out-look}{%
\chapter{Outlook}\label{out-look}}

Here goes some texts.

\hypertarget{machine-learning}{%
\section{Machine Learning}\label{machine-learning}}

Here goes some texts.

  \bibliography{book.bib,packages.bib}

\end{document}
