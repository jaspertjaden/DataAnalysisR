% Options for packages loaded elsewhere
\PassOptionsToPackage{unicode}{hyperref}
\PassOptionsToPackage{hyphens}{url}
%
\documentclass[
]{book}
\usepackage{amsmath,amssymb}
\usepackage{lmodern}
\usepackage{iftex}
\ifPDFTeX
  \usepackage[T1]{fontenc}
  \usepackage[utf8]{inputenc}
  \usepackage{textcomp} % provide euro and other symbols
\else % if luatex or xetex
  \usepackage{unicode-math}
  \defaultfontfeatures{Scale=MatchLowercase}
  \defaultfontfeatures[\rmfamily]{Ligatures=TeX,Scale=1}
\fi
% Use upquote if available, for straight quotes in verbatim environments
\IfFileExists{upquote.sty}{\usepackage{upquote}}{}
\IfFileExists{microtype.sty}{% use microtype if available
  \usepackage[]{microtype}
  \UseMicrotypeSet[protrusion]{basicmath} % disable protrusion for tt fonts
}{}
\makeatletter
\@ifundefined{KOMAClassName}{% if non-KOMA class
  \IfFileExists{parskip.sty}{%
    \usepackage{parskip}
  }{% else
    \setlength{\parindent}{0pt}
    \setlength{\parskip}{6pt plus 2pt minus 1pt}}
}{% if KOMA class
  \KOMAoptions{parskip=half}}
\makeatother
\usepackage{xcolor}
\usepackage{longtable,booktabs,array}
\usepackage{calc} % for calculating minipage widths
% Correct order of tables after \paragraph or \subparagraph
\usepackage{etoolbox}
\makeatletter
\patchcmd\longtable{\par}{\if@noskipsec\mbox{}\fi\par}{}{}
\makeatother
% Allow footnotes in longtable head/foot
\IfFileExists{footnotehyper.sty}{\usepackage{footnotehyper}}{\usepackage{footnote}}
\makesavenoteenv{longtable}
\usepackage{graphicx}
\makeatletter
\def\maxwidth{\ifdim\Gin@nat@width>\linewidth\linewidth\else\Gin@nat@width\fi}
\def\maxheight{\ifdim\Gin@nat@height>\textheight\textheight\else\Gin@nat@height\fi}
\makeatother
% Scale images if necessary, so that they will not overflow the page
% margins by default, and it is still possible to overwrite the defaults
% using explicit options in \includegraphics[width, height, ...]{}
\setkeys{Gin}{width=\maxwidth,height=\maxheight,keepaspectratio}
% Set default figure placement to htbp
\makeatletter
\def\fps@figure{htbp}
\makeatother
\setlength{\emergencystretch}{3em} % prevent overfull lines
\providecommand{\tightlist}{%
  \setlength{\itemsep}{0pt}\setlength{\parskip}{0pt}}
\setcounter{secnumdepth}{5}
\usepackage{booktabs}
\ifLuaTeX
  \usepackage{selnolig}  % disable illegal ligatures
\fi
\usepackage[]{natbib}
\bibliographystyle{plainnat}
\IfFileExists{bookmark.sty}{\usepackage{bookmark}}{\usepackage{hyperref}}
\IfFileExists{xurl.sty}{\usepackage{xurl}}{} % add URL line breaks if available
\urlstyle{same} % disable monospaced font for URLs
\hypersetup{
  pdftitle={Data Analysis with R for Social Scientists},
  pdfauthor={Jakob Tures \& Jasper Tjaden},
  hidelinks,
  pdfcreator={LaTeX via pandoc}}

\title{Data Analysis with R for Social Scientists}
\author{Jakob Tures \& Jasper Tjaden}
\date{2023-08-04}

\begin{document}
\maketitle

{
\setcounter{tocdepth}{1}
\tableofcontents
}
\hypertarget{intro}{%
\chapter*{Intro}\label{intro}}
\addcontentsline{toc}{chapter}{Intro}

This course offers an accessible and easy introduction to one of the fastest growing statistical packages used in social science and data science more generally.

Please download the data used in the course \href{https://www.worldvaluessurvey.org/WVSDocumentationWV7.jsp}{here}. To find more about me, have a look at \href{https://jaspertjaden.com}{my website}. Also, feel free to watch me as I walk you through each lesson \href{https://www.youtube.com/playlist?list=PLr43hk2e3hFMg4tZdJsN0qzG5YkQB3A1c}{here}.

\textbf{Overview over the Course :}

\begin{itemize}
\tightlist
\item
  \textbf{\protect\hyperlink{intro-sem}{Week 1: Introduction to Seminar}}
\item
  \textbf{\protect\hyperlink{eda-1}{Week 2: Exploratory Data Analysis-I}}
\item
  \textbf{\protect\hyperlink{eda-2}{Week 3: Exploratory Data Analysis-II}}
\item
  \textbf{\protect\hyperlink{lin-t}{Week 4: Linear Regression-Theory}}
\item
  \textbf{\protect\hyperlink{lin-a}{Week 5: Linear Regression-Applied}}
\item
  \textbf{\protect\hyperlink{lin-e}{Week 6: Linear Regression-Exercises}}
\item
  \textbf{\protect\hyperlink{log-t}{Week 7: Logistic Regression-Theory}}
\item
  \textbf{\protect\hyperlink{log-a}{Week 8: Logistic Regression-Applied}}
\item
  \textbf{\protect\hyperlink{log-e}{Week 9: Logistic Regression-Exercises}}
\item
  \textbf{\protect\hyperlink{pm-t}{Week 10: Prediction or Margins-Theory}}
\item
  \textbf{\protect\hyperlink{pm-e}{Week 11: Prediction or Margins-Exercises}}
\item
  \textbf{\protect\hyperlink{report-v}{Week 12: Reporting and Visualizing}}
\item
  \textbf{\protect\hyperlink{dis-paper}{Week 13: Discussion of Ideas and term papers}}
\item
  \textbf{\protect\hyperlink{out-look}{Week 14: Outlook}}
\end{itemize}

\hypertarget{intro-sem}{%
\chapter{Introduction to Seminar}\label{intro-sem}}

All chapters start with a first-level heading followed by your chapter title, like the line above. There should be only one first-level heading (\texttt{\#}) per .Rmd file.

\hypertarget{a-section}{%
\section{A section}\label{a-section}}

All chapter sections start with a second-level (\texttt{\#\#}) or higher heading followed by your section title, like the sections above and below here. You can have as many as you want within a chapter.

\hypertarget{an-unnumbered-section}{%
\subsection*{An unnumbered section}\label{an-unnumbered-section}}
\addcontentsline{toc}{subsection}{An unnumbered section}

Chapters and sections are numbered by default. To un-number a heading, add a \texttt{\{.unnumbered\}} or the shorter \texttt{\{-\}} at the end of the heading, like in this section.

\hypertarget{eda-1}{%
\chapter{Exloratory Data Analysis - I}\label{eda-1}}

Here goes some texts.

\hypertarget{load-data}{%
\section{Load data}\label{load-data}}

Here goes some texts.

\hypertarget{introduce-wvs}{%
\section{Introduce WVS}\label{introduce-wvs}}

Here goes some texts.

\hypertarget{glimpse}{%
\section{glimpse()}\label{glimpse}}

Here goes some texts.

\hypertarget{skim}{%
\section{skim()}\label{skim}}

Here goes some texts.

\hypertarget{types-of-variables-skalen}{%
\section{Types of Variables/ Skalen}\label{types-of-variables-skalen}}

Here goes some texts.

\hypertarget{univaraite-statistics-means-sds-min-max}{%
\section{Univaraite statistics (means, SDs, min, max)}\label{univaraite-statistics-means-sds-min-max}}

Here goes some texts.

\hypertarget{ggplot}{%
\section{ggplot}\label{ggplot}}

Here goes some texts.

\hypertarget{histograms}{%
\subsection{Histograms}\label{histograms}}

Here goes some texts.

\hypertarget{boxplots}{%
\subsection{Boxplots}\label{boxplots}}

Here goes some texts.

\hypertarget{bar-graphs}{%
\subsection{Bar graphs}\label{bar-graphs}}

Here goes some texts.

\hypertarget{scatterplots}{%
\subsection{Scatterplots}\label{scatterplots}}

Here goes some texts.

\hypertarget{gtsummary}{%
\section{gtsummary}\label{gtsummary}}

Here goes some texts.

\hypertarget{kreuztabellen}{%
\subsection{Kreuztabellen}\label{kreuztabellen}}

Here goes some texts.

\hypertarget{eda-2}{%
\chapter{Exloratory Data Analysis - II}\label{eda-2}}

Here goes some texts.

\hypertarget{markdown-introduction}{%
\section{Markdown Introduction}\label{markdown-introduction}}

Here goes some texts.

\hypertarget{applying-edawvsown-data}{%
\section{Applying EDA(WVS/own data)}\label{applying-edawvsown-data}}

Here goes some texts.

\hypertarget{lin-t}{%
\chapter{Linear Regression - Theory}\label{lin-t}}

\hypertarget{what-is-it}{%
\section{What is it?}\label{what-is-it}}

Here goes some texts.

\hypertarget{when-and-for-what-it-can-used}{%
\section{When and for what it can used?}\label{when-and-for-what-it-can-used}}

Here goes some texts.

\hypertarget{formula-short}{%
\section{Formula (short)}\label{formula-short}}

Here goes some texts.

\hypertarget{assumptions-short}{%
\section{Assumptions (short)}\label{assumptions-short}}

Here goes some texts.

\#\#Interpretation of results
Here goes some texts.

\hypertarget{mediation}{%
\section{Mediation}\label{mediation}}

Maybe theory into DAG session and example into application?

\hypertarget{interactions}{%
\section{Interactions?}\label{interactions}}

Here goes some texts.

\hypertarget{multiple-outcomes}{%
\section{Multiple outcomes}\label{multiple-outcomes}}

Here goes some texts.

\hypertarget{lin-a}{%
\chapter{Linear Regression - Applied}\label{lin-a}}

\hypertarget{incl.-short-theory-and-dag}{%
\section{Incl. Short Theory and DAG}\label{incl.-short-theory-and-dag}}

Here goes some texts.

\hypertarget{application-with-wvs-data}{%
\subsection{Application with WVS data}\label{application-with-wvs-data}}

Here goes some texts.

\hypertarget{interpreation-of-regression-tables-in-practice}{%
\section{Interpreation of regression tables in practice}\label{interpreation-of-regression-tables-in-practice}}

Here goes some texts.

\hypertarget{mediation-1}{%
\section{Mediation}\label{mediation-1}}

Here goes some texts.

\hypertarget{total-and-direct-effect}{%
\subsection{Total and Direct effect}\label{total-and-direct-effect}}

Here goes some texts.

\hypertarget{regressional-diagnostics}{%
\section{Regressional Diagnostics}\label{regressional-diagnostics}}

(Maybe)

\hypertarget{lin-e}{%
\chapter{Linear Regression - Exercises}\label{lin-e}}

Here goes some texts.

\hypertarget{application-of-linear-regression}{%
\section{Application of Linear Regression}\label{application-of-linear-regression}}

With WVS/own data: Students apply linear regression.

\hypertarget{log-t}{%
\chapter{Logistic Regression - Theory}\label{log-t}}

\hypertarget{what-is-it-1}{%
\section{What is it?}\label{what-is-it-1}}

Here goes some texts.

\hypertarget{when-and-for-what-it-can-used-1}{%
\section{When and for what it can used?}\label{when-and-for-what-it-can-used-1}}

Here goes some texts.

\hypertarget{formula-short-1}{%
\section{Formula (short)}\label{formula-short-1}}

Here goes some texts.

\hypertarget{assumptions-short-1}{%
\section{Assumptions (short)}\label{assumptions-short-1}}

Here goes some texts.

\#\#Interpretation of results
Here goes some texts.

\hypertarget{mediation-2}{%
\section{Mediation}\label{mediation-2}}

Maybe theory into DAG session and example into application?

\hypertarget{multiple-outcomes-1}{%
\section{Multiple outcomes}\label{multiple-outcomes-1}}

Here goes some texts.

\hypertarget{multinomial}{%
\section{Multinomial}\label{multinomial}}

\hypertarget{log-a}{%
\chapter{Logistic Regression - Applied}\label{log-a}}

\hypertarget{incl.-short-theory-and-dag-1}{%
\section{Incl. Short Theory and DAG}\label{incl.-short-theory-and-dag-1}}

Here goes some texts.

\hypertarget{application-with-wvs-data-1}{%
\subsection{Application with WVS data}\label{application-with-wvs-data-1}}

Here goes some texts.

\hypertarget{interpreation-of-regression-tables-in-practice-1}{%
\section{Interpreation of regression tables in practice}\label{interpreation-of-regression-tables-in-practice-1}}

Here goes some texts.

\hypertarget{mediation-3}{%
\section{Mediation}\label{mediation-3}}

Here goes some texts.

\hypertarget{total-and-direct-effect-1}{%
\subsection{Total and Direct effect}\label{total-and-direct-effect-1}}

Here goes some texts.

\hypertarget{regressional-diagnostics-1}{%
\section{Regressional Diagnostics}\label{regressional-diagnostics-1}}

(Maybe)

\hypertarget{log-e}{%
\chapter{Logistic Regression - Exercises}\label{log-e}}

Here goes some texts.

\hypertarget{application-of-logistic-regression}{%
\section{Application of Logistic Regression}\label{application-of-logistic-regression}}

With WVS/own data: Students apply linear regression.

\hypertarget{pm-t}{%
\chapter{Prediction or Margins - Theory}\label{pm-t}}

Here goes some texts.

\hypertarget{predicted-probabilities}{%
\section{Predicted probabilities}\label{predicted-probabilities}}

At various co-variate levels

\hypertarget{marginal-effects}{%
\section{Marginal Effects}\label{marginal-effects}}

\hypertarget{pm-e}{%
\chapter{Prediction or Margins - Exercises}\label{pm-e}}

Here goes some texts.

\hypertarget{application-of-regression}{%
\section{Application of Regression}\label{application-of-regression}}

With WVS/own data: Students apply linear+logistic regression from previous exercises.

\hypertarget{report-v}{%
\chapter{Reporting and Visualization}\label{report-v}}

Here goes some texts.

\hypertarget{formatted-regression-tables}{%
\section{Formatted regression tables}\label{formatted-regression-tables}}

Here goes some texts.

\hypertarget{publication-ready-formatting-labelling-of-visuals}{%
\section{Publication-ready formatting/ labelling of visuals}\label{publication-ready-formatting-labelling-of-visuals}}

Here goes some texts.

\hypertarget{coefficient-plots}{%
\section{coefficient plots}\label{coefficient-plots}}

Here goes some texts.

\hypertarget{dis-paper}{%
\chapter{Discussion of ideas and term papers}\label{dis-paper}}

Here goes some texts.

\hypertarget{out-look}{%
\chapter{Outlook}\label{out-look}}

Here goes some texts.

\hypertarget{machine-learning}{%
\section{Machine Learning}\label{machine-learning}}

Here goes some texts.

  \bibliography{book.bib,packages.bib}

\end{document}
